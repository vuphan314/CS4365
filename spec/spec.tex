\documentclass[letterpaper, 12pt]{extarticle}
% The sizes available are 8pt, 9pt, 10pt, 11pt, 12pt, 14pt, 17pt, and 20pt.

\usepackage[scale=0.9]{geometry}

\usepackage{scrextend}

\setlength{\parindent}{0pt}

\setlength{\parskip}{10pt}

\usepackage{enumitem}
\setlist[itemize]{topsep=0pt, parsep=0pt, itemsep=0pt}

\usepackage{authblk}

\usepackage{natbib}
\bibliographystyle{plainnat}

\usepackage[bookmarksnumbered]{hyperref}

\usepackage{enumitem}
\setlist[enumerate]{label*=\arabic*.}

\usepackage[yyyymmdd]{datetime}

\usepackage{amsmath}
\usepackage{amsthm}
\usepackage{amssymb}

\theoremstyle{definition} % non-italic
\newtheorem{theorem}{Theorem}[subsection]
\newtheorem{lemma}[theorem]{Lemma}
\newtheorem{proposition}[theorem]{Proposition}
\newtheorem{corollary}[theorem]{Corollary}
\newtheorem{remark}[theorem]{Remark}
\newtheorem{definition}[theorem]{Definition}
\newtheorem{example}[theorem]{Example}

\usepackage[none]{hyphenat}

\usepackage{fancyhdr}
\pagestyle{fancy}

\fancyhf{}
\renewcommand{\headrulewidth}{0pt}

\usepackage{lastpage}
\rfoot{Page \thepage{} of \pageref{LastPage}}

%%%%%%%%%%%%%%%%%%%%%%%%%%%%%%%%%%%%%%%%%%%%%%%%%%%%%%%%%%%%

%:commands

\newcommand{\textdef}[1]{\textbf{#1}}
\newcommand{\code}[1]{\texttt{#1}}
\newcommand{\kn}{\code{Knotty}}

%%%%%%%%%%%%%%%%%%%%%%%%%%%%%%%%%%%%%%%%%%%%%%%%%%%%%%%%%%%%

% environments

\newenvironment{codeblock}
    {\begin{addmargin}{0.5in} \ttfamily}
    {\end{addmargin} \par}

%%%%%%%%%%%%%%%%%%%%%%%%%%%%%%%%%%%%%%%%%%%%%%%%%%%%%%%%%%%%

\title{The \kn{} Companion}

\author{Vu Phan}

\begin{document}

\maketitle

\begin{abstract}
This document is the specification of the \kn{} language.
\end{abstract}

\tableofcontents

\thispagestyle{fancy}

%%%%%%%%%%%%%%%%%%%%%%%%%%%%%%%%%%%%%%%%%%%%%%%%%%%%%%%%%%%%

\section{Syntax}

%%%%%%%%%%%%%%%%%%%%%%%%%%%%%%%%%%%%%%%%%%%%%%%%%%%%%%%%%%%%

\subsection{Lexicon}

These lexemes are \textdef{keywords}:
\begin{codeblock}
unknown constant function let return check \\
if else true false or and not
\end{codeblock}

A \textdef{number} is either:
\begin{itemize}
\item \code{i}, or
\item one or more consecutive digits
(\code{0}-\code{9})
\end{itemize}

An \textdef{identifier} is one letter
(\code{a}-\code{z} or \code{A}-\code{Z})
followed by zero or more letters and digits.
Also, an identifier must not be a keyword.

Note: blank characters (spaces, tabs, newlines)
are delimiters.

%%%%%%%%%%%%%%%%%%%%%%%%%%%%%%%%%%%%%%%%%%%%%%%%%%%%%%%%%%%%

\subsection{Grammar}

\subsubsection{Names}

An \textdef{unknown name} is an identifier.
So is a \textdef{constant name}, a \textdef{function name},
a \textdef{temporary name}, and a \textdef{check name}.

\subsubsection{Named Terms}

A \textdef{named term} is an unknown name, constant name,
function name, formal parameter, temporary name, or
actual function expression.

%%%%%%%%%%%%%%%%%%%%%%%%%%%%%%%%%%%%%%%%%%%%%%%%%%%%%%%%%%%%

\section{Semantics}

%%%%%%%%%%%%%%%%%%%%%%%%%%%%%%%%%%%%%%%%%%%%%%%%%%%%%%%%%%%%

\end{document}
